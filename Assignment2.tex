\documentclass[journal,12pt,twocolumn]{IEEEtran}

\usepackage{setspace}
\usepackage{gensymb}
\singlespacing
\usepackage[cmex10]{amsmath}

\usepackage{amsthm}

\usepackage{mathrsfs}
\usepackage{txfonts}
\usepackage{stfloats}
\usepackage{bm}
\usepackage{cite}
\usepackage{cases}
\usepackage{subfig}

\usepackage{longtable}
\usepackage{multirow}

\usepackage{enumitem}
\usepackage{mathtools}
\usepackage{steinmetz}
\usepackage{tikz}
\usepackage{circuitikz}
\usepackage{verbatim}
\usepackage{tfrupee}
\usepackage[breaklinks=true]{hyperref}
\usepackage{graphicx}
\usepackage{tkz-euclide}

\usetikzlibrary{calc,math}
\usepackage{listings}
    \usepackage{color}                                            %%
    \usepackage{array}                                            %%
    \usepackage{longtable}                                        %%
    \usepackage{calc}                                             %%
    \usepackage{multirow}                                         %%
    \usepackage{hhline}                                           %%
    \usepackage{ifthen}                                           %%
    \usepackage{lscape}     
\usepackage{multicol}
\usepackage{chngcntr}

\DeclareMathOperator*{\Res}{Res}

\renewcommand\thesection{\arabic{section}}
\renewcommand\thesubsection{\thesection.\arabic{subsection}}
\renewcommand\thesubsubsection{\thesubsection.\arabic{subsubsection}}

\renewcommand\thesectiondis{\arabic{section}}
\renewcommand\thesubsectiondis{\thesectiondis.\arabic{subsection}}
\renewcommand\thesubsubsectiondis{\thesubsectiondis.\arabic{subsubsection}}


\hyphenation{op-tical net-works semi-conduc-tor}
\def\inputGnumericTable{}                                 %%

\lstset{
%language=C,
frame=single, 
breaklines=true,
columns=fullflexible
}
\begin{document}


\newtheorem{theorem}{Theorem}[section]
\newtheorem{problem}{Problem}
\newtheorem{proposition}{Proposition}[section]
\newtheorem{lemma}{Lemma}[section]
\newtheorem{corollary}[theorem]{Corollary}
\newtheorem{example}{Example}[section]
\newtheorem{definition}[problem]{Definition}

\newcommand{\BEQA}{\begin{eqnarray}}
\newcommand{\EEQA}{\end{eqnarray}}
\newcommand{\define}{\stackrel{\triangle}{=}}
\bibliographystyle{IEEEtran}
\raggedbottom
\setlength{\parindent}{0pt}
\providecommand{\mbf}{\mathbf}
\providecommand{\pr}[1]{\ensuremath{\Pr\left(#1\right)}}
\providecommand{\qfunc}[1]{\ensuremath{Q\left(#1\right)}}
\providecommand{\sbrak}[1]{\ensuremath{{}\left[#1\right]}}
\providecommand{\lsbrak}[1]{\ensuremath{{}\left[#1\right.}}
\providecommand{\rsbrak}[1]{\ensuremath{{}\left.#1\right]}}
\providecommand{\brak}[1]{\ensuremath{\left(#1\right)}}
\providecommand{\lbrak}[1]{\ensuremath{\left(#1\right.}}
\providecommand{\rbrak}[1]{\ensuremath{\left.#1\right)}}
\providecommand{\cbrak}[1]{\ensuremath{\left\{#1\right\}}}
\providecommand{\lcbrak}[1]{\ensuremath{\left\{#1\right.}}
\providecommand{\rcbrak}[1]{\ensuremath{\left.#1\right\}}}
\theoremstyle{remark}
\newtheorem{rem}{Remark}
\newcommand{\sgn}{\mathop{\mathrm{sgn}}}
\providecommand{\abs}[1]{\left\vert#1\right\vert}
\providecommand{\res}[1]{\Res\displaylimits_{#1}} 
\providecommand{\norm}[1]{\left\lVert#1\right\rVert}
%\providecommand{\norm}[1]{\lVert#1\rVert}
\providecommand{\mtx}[1]{\mathbf{#1}}
\providecommand{\mean}[1]{E\left[ #1 \right]}
\providecommand{\fourier}{\overset{\mathcal{F}}{ \rightleftharpoons}}
%\providecommand{\hilbert}{\overset{\mathcal{H}}{ \rightleftharpoons}}
\providecommand{\system}{\overset{\mathcal{H}}{ \longleftrightarrow}}
	%\newcommand{\solution}[2]{\textbf{Solution:}{#1}}
\newcommand{\solution}{\noindent \textbf{Solution: }}
\newcommand{\cosec}{\,\text{cosec}\,}
\providecommand{\dec}[2]{\ensuremath{\overset{#1}{\underset{#2}{\gtrless}}}}
\newcommand{\myvec}[1]{\ensuremath{\begin{pmatrix}#1\end{pmatrix}}}
\newcommand{\mydet}[1]{\ensuremath{\begin{vmatrix}#1\end{vmatrix}}}
\numberwithin{equation}{subsection}
\makeatletter
\@addtoreset{figure}{problem}
\makeatother
\let\StandardTheFigure\thefigure
\let\vec\mathbf
\renewcommand{\thefigure}{\theproblem}
\def\putbox#1#2#3{\makebox[0in][l]{\makebox[#1][l]{}\raisebox{\baselineskip}[0in][0in]{\raisebox{#2}[0in][0in]{#3}}}}
     \def\rightbox#1{\makebox[0in][r]{#1}}
     \def\centbox#1{\makebox[0in]{#1}}
     \def\topbox#1{\raisebox{-\baselineskip}[0in][0in]{#1}}
     \def\midbox#1{\raisebox{-0.5\baselineskip}[0in][0in]{#1}}
\vspace{3cm}
\title{Assignment 2}
\author{P Ganesh Nikhil Madhav -CS20BTECH11036}
\maketitle
\newpage
\bigskip
\renewcommand{\thefigure}{\theenumi}
\renewcommand{\thetable}{\theenumi}
Download  latex-tikz codes from 
\begin{lstlisting}
https://github.com/Nik123-cpp/Assignment-2/blob/main/Assignment2.tex
\end{lstlisting}
\section{Gate Problem 64}
Let $X_{n}$ denote the sum of points obtained when n fair dice are rolled together.The Expectation and Variance of $X_{n}$ are
\section{Solution}
$X_{n}$ denote the random variable which denotes the sum of points obtained when n fair dice are rolled together
We know, when one dice is rolled probability i.e \pr{X_{1}=r} for all r in \{1,2,3,4,5,6\} is equal to  $\frac{1}{6}$ ,Now i will calculate expectation value
using below formula;
\begin{align}
 E( X_{1})&=  \sum_{r=1}^6 r.\pr{X_{1}=r}
 \label{eq:eq1}
\\
&=\frac{1}{6}\sum_{r=1}^6 .r
\\
&=\frac{1}{6}.\frac{6(7)}{2} =\frac{7}{2}.
\end{align}
\begin{enumerate}
\item similarly when n dice are rolled ,expectation value from each dice 
= $\frac{7}{2} $. 
So expectation of sum of points on n dice is n times the expectation value from each dice i.e
\begin{align}
    E(X_{n})& = n(E(X_{1})) =\frac{7}{2} n 
\label{eq:eq2}
\end{align}
\item By Using the following formula and using \eqref{eq:eq2}  we can calculate variance of  $X_{n}$ 
    \begin{align}
    V(X_{n})&=(E(X_{n})^{2}) - (E(X_{n}))^{2}
    \\
    &=(E(X_{n})^{2}) - (\frac{49}{4}n^{2})
    \label{eq:eq3}
    \end{align}
    In \eqref{eq:eq3} ,By substituting n as 1 
    \begin{align}
        V(X_{1})&=(E(X_{1})^{2}) - \frac{49}{4}
        \label{eq:eq4}
    \end{align}
    Now calculating E($X_{1}^{2}$),
    \begin{align}
      E(X_{1}^{2})&=\sum_{r=1}^6r^{2}.\frac{1}{6}
      \\
      &=\frac{1}{6}.\frac{6(6+1)(2(6)+1)}{6}
      \\
      &=\frac{1}{6}.\frac{6.7.13}{6}
      \\
      &=\frac{7.13}{6}
      \\
      &=\frac{91}{6}
      \label{eq:eq5}
    \end{align}
    From \eqref{eq:eq4} and \eqref{eq:eq5}
    \begin{align}
        V(X_{1})&=\frac{91}{6} -\frac{49}{4}
        \\
        &=\frac{182-147}{12}
        \\
        &=\frac{35}{12}
        \label{eq:eq6}
    \end{align}
    Since the variance of a sum of independent random variables is the sum of their variances. So, When n dice are rolled the variance of $X_{n}$ is n times the variance of the value when one dice is rolled i.e
    \begin{align}
        V(X_{n})&=n.V(X_{1}) = \frac{35}{12}n
    \end{align}
    Hence option(B) is correct.
\end{enumerate}
\end{document}
